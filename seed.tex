%%============================ Compiler Directives =======================%%
%%                                                                        %%
% !TeX program = lualatex
% !TeX encoding = utf8
% !TeX spellcheck = uk_UA
%%                                                                        %%
%%============================== Клас документа ==========================%%
%%                                                                        %%
\documentclass[]{article}
\usepackage[fontsize = 12pt]{fontsize}
\usepackage{ifluatex}
%%                                                                        %%
%%========================== Мови, шрифти та кодування ===================%%
%%
\ifluatex                                                                        %%
	\usepackage{fontspec}
	\setsansfont{CMU Sans Serif}%{Arial}
	\setmainfont{CMU Serif}%{Times New Roman}
	\setmonofont{CMU Typewriter Text}%{Consolas}
	\defaultfontfeatures{Ligatures={TeX}}
	\usepackage[math-style=TeX]{unicode-math}
\else
	\usepackage[utf8]{inputenc}
	\usepackage[T2A,T1]{fontenc}
	\usepackage{amsmath}
	%\usepackage{pscyr}
	\usepackage{cmap}
\fi
\usepackage[english, russian, ukrainian]{babel}
%%                                                                        %%
%%============================= Геометрія сторінки =======================%%
%%                                                                        %%
\usepackage[%
	a4paper,%
	footskip=1cm,%
	headsep=0.3cm,%
	top=2cm, %поле сверху
	bottom=2cm, %поле снизу
	left=2cm, %поле ліворуч
	right=2cm, %поле праворуч
    ]{geometry}
%%                                                                        %%
%%============================== Інтерліньяж  ============================%%
%%                                                                        %%
\renewcommand{\baselinestretch}{1}
%-------------------------  Подавление висячих строк  --------------------%%
\clubpenalty =10000
\widowpenalty=10000
%---------------------------------Інтервали-------------------------------%%
\setlength{\parskip}{0.5ex}%
\setlength{\parindent}{2.5em}%
%%                                                                        %%
%%=========================== Математичні пакети і графіка ===============%%
%%                                                                        %%
\usepackage{tikz, pgfplots}
\usepackage{subcaption, xcolor}
%%                                                                        %%
%%========================== Гіперпосилення (href) =======================%%
%%                                                                        %%
\usepackage[%colorlinks=true,
	%urlcolor = blue, %Colour for external hyperlinks
	%linkcolor  = malina, %Colour of internal links
	%citecolor  = green, %Colour of citations
	bookmarks = true,
	bookmarksnumbered=true,
	unicode,
	linktoc = all,
	hypertexnames=false,
	pdftoolbar=false,
	pdfpagelayout=TwoPageRight,
	pdfauthor={Ponomarenko S.M. aka sergiokapone},
	pdfdisplaydoctitle=true,
	pdfencoding=auto
	]%
	{hyperref}
		\makeatletter
	\AtBeginDocument{
	\hypersetup{
		pdfinfo={
		Title={\@title},
		}
	}
	}
	\makeatother
%%                                                                        %%
%%============================ Заголовок та автори =======================%%
%%                                                                        %%
\title{}
\author{}
%%                                                                        %%
%%========================================================================%%

\pgfset{
    declare function = {
        gauss(\m,\s,\x) = 1/(sqrt(2*pi)*\s)*exp(-0.5*(\x-\m)^2/\s^2);
        %                    valuegauss(\x,\m,\s) = 1/(sqrt(2*pi)*\s)*exp(-0.5*(\x-\m)^2/\s^2);
    },
}

\pgfplotsset{%
    InfoPlotGrid/.style={%
        axis lines = left,
        extra tick style={% changes for all extra ticks
            %						tick align=outside,
            major grid style={dashed,draw=red}
        },
        extra x tick style={
            major tick style={
                gray,
            },
            tick label style={
                /pgf/number format/.cd, fixed, fixed zerofill,
                precision=2,
            }
        },
        extra y tick style={% changes for extra y ticks
            major tick style={
                gray,
            },
        },
        xtick = \empty,
        xmajorgrids=true,
        ymajorgrids=true,
        every axis x label/.style={at={(current axis.right of origin)},anchor=north},
    },
}

\AtBeginDocument{%
    \let\phi\varphi
    \let\epsilon\varepsilon
    \DeclareMathOperator{\const}{const}
    \newcommand{\sn}{\sum\limits_{i = 1}^n}
    \newcommand{\average}[1]{\left\langle #1 \right\rangle}
}

\begin{document}
% 7 21 30
\foreach \i in {2300} {

\pgfmathsetseed{\i}%2300

\begin{center}
    \i
\end{center}

        \begin{tikzpicture}[scale=0.3]
            \pgfplotsset{compat=1.8}
            \begin{axis}[InfoPlotGrid, every node/.style={transform shape}
                axis y discontinuity=none,
                width=\linewidth,
                ytick distance = 0.1,
                xtick distance = 1,
                xlabel = {$x$},
                ylabel = {Частка та $w(x)$},
                ymax=0.6,
                extra x ticks={12.8},
                extra x tick labels={$\average{x}$},
                ]
                \addplot [ultra thick, ybar interval, bar width=0.2, domain=8:17, samples=15,]
                (x, {1.5*rnd*gauss(12.5, 1,x)});
                \addplot [domain=8:17, samples=200, smooth, cyan!80!blue, thick]
                (x, {gauss(12.5, 1,x)});

            \end{axis}
        \end{tikzpicture}
        %
        \begin{tikzpicture}[scale=0.3]
            \pgfplotsset{compat=1.8}
            \begin{axis}[InfoPlotGrid,  every node/.style={transform shape}
                axis y discontinuity=none,
                width=\linewidth,
                ytick distance = 0.1,
                xtick distance = 1,
                xlabel = {$x$},
                ylabel = {Частка та $w(x)$},
                ymax=0.6,
                extra x ticks={12.5},
                extra x tick labels={$\average{x}$},
                ]
                \addplot [domain=8:17, samples=100, ybar interval]
                (x, {1.5*rnd*gauss(12.5, 1,x)});

                \addplot [domain=8:17, samples=200, smooth, cyan!80!blue, thick]
                (x, {gauss(12.5, 1,x)});
            \end{axis}
        \end{tikzpicture}
}

\clearpage

\pgfmathsetseed{2300}%2300
\begin{tikzpicture}
    \pgfplotsset{compat=1.8}
    \begin{axis}[InfoPlotGrid,  every node/.style={transform shape}
        axis y discontinuity=none,
        width=\linewidth,
        ytick distance = 0.1,
        xtick distance = 1,
        xlabel = {$x$},
        ylabel = {},
        ymax=6,
        extra x ticks={12.5},
        extra x tick labels={$\average{x}$},
        ]
        \addplot [domain=8:17, samples=20, ybar interval]
        (x, {10*gauss(12.5, 1,x)});

    \end{axis}
\end{tikzpicture}



\end{document}


